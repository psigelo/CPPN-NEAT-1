\subsection*{Compiling}

Just execute the {\itshape makefile} in order to compile the program

``` \$ cd C\-P\-P\-N-\/\-N\-E\-A\-T \$ make clean \$ make ```

\subsection*{Usage}

We've incorporated both of the minimal actions related to evolving a network, i.\-e. training a certain network \& using it.

\subsubsection*{Training a network}

In order to rain a network ( by default {\itshape genetic\-\_\-encodings/default\-\_\-00.\-genetic\-\_\-encoding} ) you must use the flag $\ast$-\/t$\ast$ followed by the file to train

``` \$ cd C\-P\-P\-N-\/\-N\-E\-A\-T \$ ./executables/\-C\-P\-P\-N-\/\-N\-E\-A\-T -\/t genetic\-\_\-encodings/default\-\_\-00.\-genetic\-\_\-encoding ```

This will print in the console the

This will create a file in the same folder with the same name ( with extension $\ast$$\ast$.out.\-genetic\-\_\-encoding$\ast$$\ast$ ) which contain the network structure.


\begin{DoxyItemize}
\item We use J\-S\-O\-N as a network structure in order to make a visual panel in the future using {\itshape html} and {\itshape jvascript}.
\end{DoxyItemize}

\subsubsection*{Use a trained network}

``` \$ cd C\-P\-P\-N-\/\-N\-E\-A\-T \$ \$ \# In order to train a network file (extension .genetic\-\_\-encoding) \$ ./executables/\-C\-P\-P\-N-\/\-N\-E\-A\-T -\/t genetic\-\_\-encodings/default\-\_\-00.\-genetic\-\_\-encoding \$ \$ \# In order to execute the trained network \$ ./executables/\-C\-P\-P\-N-\/\-N\-E\-A\-T -\/x genetic\-\_\-encodings/default\-\_\-00.\-out.\-genetic\-\_\-encoding -\/v 0 0 ``` 